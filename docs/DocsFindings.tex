\documentclass[letterpaper, 11 pt]{book}

\usepackage{amsmath}  % needed for \tfrac, \bmatrix, etc.
\usepackage{amsfonts} % needed for bold Greek, Fraktur, and blackboard bold
\usepackage{graphicx} % needed for figures
\usepackage{caption} 
\usepackage{hyperref}
\usepackage{float}
\usepackage{subcaption}
\setlength{\textwidth}{6.5in}
\setlength{\textheight}{8.5in}
\setlength{\oddsidemargin}{0pt}
\setlength{\evensidemargin}{0pt}
\setlength{\topmargin}{0pt}
\setlength{\marginparsep}{0pt}
\setlength{\marginparwidth}{1in}

% -------------------------------------------------------------------------------------
% BEGIN DOCUMENT
% -------------------------------------------------------------------------------------
\begin{document}
\title{Single Photon Experiments 2023}
\author{Tai Xiang}
\maketitle
\pagestyle{empty}

% -------------------------------------------------------------------------------------
% TABLE OF CONTENTS
% -------------------------------------------------------------------------------------
\tableofcontents
\newpage

% -------------------------------------------------------------------------------------
% INTRODUCTION
% -------------------------------------------------------------------------------------
\section{Summary}
\newpage

\section{Single Photon Detection Set-up}

The basic set-up involves a helium-neon laser in the IR band driven into a BBO crystal. Photons that enter the BBO crystal from the laser
undergo spontaneous parametric down-conversion and are then detected within two photon detectors, as shown in figure \ref{fig:spcm_setup_basic}.
\begin{figure}[H]
    \centering
    \includegraphics[width = 7cm]{setup_basic.png}
    \caption{Basic set-up for the SPCM experiment.}
    \label{fig:spcm_setup_basic}
\end{figure}

Additional beam-splitters can be added between the BBO and the detectors to
yield a set-up such as the one shown in figure \ref{fig:spcm_setup_bs}.
\begin{figure}[H]
    \centering
    \includegraphics[width = 7cm]{setup.png}
    \caption{Beam-splitter set-up for the SPCM experiment.}
    \label{fig:spcm_setup_bs}
\end{figure}

This set-up will be the primary set-up used for the experiments and upgrades
listed in the later sections. 


\section{Photon Statistics and Randomness}

One of the extensions I explored with the single photon counting module was to
apply it to the task of random number generation. Exploiting the quantum
intepretation of photons traveling within a beam-splitter and the Poissonian
nature of light, a high bitrate random number generator can be developed from
the current CCM experimental set-up. The random numbers generated from this
set-up can then be used to map out password strings or generate security keys.

With the beam-splitter source of randomness, the set-up generates random bits by
observing what detector measures a photon. In the set-up shown in figure
\ref{fig:spcm_setup_bs}, a 50--50 beam-splitter is placed between the detector B
and B'. In the quantum mechnical interpretation of light, the 50--50
beam-splitter causes an incident photon to have a 50--50 chance of arriving
within detector B or B'. Assigning a bit of 0 to B and a bit of 1 to B', we then
have a 50--50 chance of getting a bit of 1 or 0. By inserting an additional
detector A' that is analogous to B and B', and then adding a 50--50
beam-splitter between A and A', another random bit can be generated. 

Furthermore, since photons are emitted from the laser in a Poissonian nature, we
can split off a Poissonian distribution fitted to the count of arriving photons,
assign buckets to each, and draw random bits in accordance to what bucket a
time-step of photon counts falls into. This illustrated in figure
\ref{fig:poisson}.
\begin{figure}[H]
    \centering
    \includegraphics[width = 8cm]{poisson_cut.png}
    \caption{Poissonian distribution segmented into regions of equal area. A bit combination is then assigned to each region.}
    \label{fig:poisson}
\end{figure}

For example, taking an experiment fitted to the Poissonian displayed in figure
\ref{fig:poisson}, if there are 40 incident photons, then we get an output
bitstring of 01, if there are 45, we output 01, and so on. A poissonian can be
drawn for each detector in the experiment.

In an experiment with two beam-splitters and four detectors utilizing each of
the ports available on the SPCM-AQ4C module, a total bitrate of 10 bits per
integration time-step can be achieved. In the generation of a password or
security key, there are a total of 10 numbers, 40 symbols, 26 lower-case
letters, and 26 upper-case letters to choose from, yielding a total of 102
possible selections. A total of 7 bits can reach every single combination of
this, with $2^7 - 102 = 28$ additional mappings that are not used and can be
overflowed to the next time-step. Each experimental time-step generates 10 bits,
so we use 7 to generate a character and overflow the additional 3 bits as well
as the additional mappings to the next time-step for efficiency. 

We validate the random bits we generate with the NIST statistical test suite for
random number generates. The NIST statistical test suite contains a series of
tests utilized to determine the randomness of input number sequences. The
results of the tests are shown in table \ref{tab:rng_test}, where a score of
$8/10$ is considered passing.% TODO: CITE NIST

\begin{table}[H]
    \centering
\begin{tabular}{|l|l|}
\hline
\textbf{Statistical Test}          & \textbf{Proportion} \\\hline
Frequency                 & 10/10      \\\hline
Block Frequency           & 10/10      \\\hline
Cumulative Sums           & 10/10      \\\hline
Longest Run               & 10/10      \\\hline
Longest Run (Rank)        & 10/10      \\\hline
Longest Run (FFT)         & 10/10      \\\hline
NonOverlapping Template   & 10/10      \\\hline
NonOverlapping Template   & 9/10       \\\hline
NonOverlapping Template   & 10/10      \\\hline
NonOverlapping Template   & 10/10      \\\hline
NonOverlapping Template   & 9/10       \\\hline
NonOverlapping Template   & 10/10      \\\hline
NonOverlapping Template   & 10/10      \\\hline
NonOverlapping Template   & 10/10      \\\hline
NonOverlapping Template   & 10/10      \\\hline
NonOverlapping Template   & 10/10      \\\hline
NonOverlapping Template   & 10/10      \\\hline
NonOverlapping Template   & 10/10      \\\hline
NonOverlapping Template   & 10/10      \\\hline
NonOverlapping Template   & 8/10      \\\hline
NonOverlapping Template   & 10/10      \\\hline
NonOverlapping Template   & 10/10      \\\hline
NonOverlapping Template   & 9/10      \\\hline
NonOverlapping Template   & 9/10      \\\hline
NonOverlapping Template   & 10/10      \\\hline
NonOverlapping Template   & 10/10      \\\hline
NonOverlapping Template   & 9/10      \\\hline
NonOverlapping Template   & 10/10      \\\hline
NonOverlapping Template   & 10/10      \\\hline
NonOverlapping Template   & 10/10      \\\hline
NonOverlapping Template   & 10/10      \\\hline
NonOverlapping Template   & 10/10      \\\hline
NonOverlapping Template   & 10/10      \\\hline
NonOverlapping Template   & 10/10      \\\hline
NonOverlapping Template   & 10/10      \\\hline
NonOverlapping Template   & 10/10      \\\hline
NonOverlapping Template   & 10/10      \\\hline
Random Excursions Variant & 10/10      \\\hline
\end{tabular}
\end{table}
\label{tab:rng_test}

The random numbers generated with our set-up pass the NIST statistical tests.
However, although the set-up performs extremely well on these tasks, it should
be noted that the passing of statistical tests is not a mathematically rigorous
definition of "randomness". The set-up merely performs well on human-defined
metrics of randomness.

\section{Scattering Reduction}
An additional improvement I have made to this experimental set-up is the
development of multiple schemes to reduce scattering from background noise
sources. This section will cover a set of optimizations and developments on this
front.

\subsection{Black Paper and Boxes}

A very basic optimization that I performed to reduce scattering was to place
long sheets of black paper along the general beam-path of the laser. This was
done to ensure scattered light from sources external to the experiment, such as
light from the laptop screen and light from the controls of the coincidence
counting module, do not interfere with the experiment and generate accidental
coincidences.

Additionally, I also added a box that fits the optical components that are along
the beam-path prior to the BBO crystal. I cut a small hole that enables the beam
to travel into the box and another small hole that enables the beam to exit. Any
scattering that arises from the components inside the box is thus contained
within it and does not leak out into the room and contaminate the photons
arriving at the detectors. 

The additional pieces of paper along the beam-path as well as the box enclosure
as shown in figure \ref{fig:paper_scatter}.
\begin{figure}[H]
    \centering
    \includegraphics[width = 8cm]{poisson_cut.png}
    \caption{Poissonian distribution segmented into regions of equal area. A bit combination is then assigned to each region.}
    \label{fig:paper_scatter}
\end{figure}

\subsection{Enclosures}

Another scattering reduction optimization I have made is to create a 3-D
printable enclosure for the detectors that reduces the chance of entry for
noisey scattered photons. The two-piece enclosure fits and snaps over the
detectors and adds a cylindrical nose to the front. The long cylindrical nose
reduces the chance of photons that do not arise from the beam-path from hitting
the detector. The CAD designs are shown in figure \ref{fig:cad_enclose}.
\begin{figure}[H]%
        \centering
        \begin{subfigure}{.4\textwidth}
        \includegraphics[width=7cm]{enclosure_1.png}
        \caption{ }
        \label{fig:side}
        \end{subfigure}
        \begin{subfigure}{.4\textwidth}
        \includegraphics[width = 7cm]{enclosure_2.png}
        \caption{ }
        \label{fig:back}
        \end{subfigure}
        \caption{Figure \ref{fig:side} displays the side view of the enclosure.
        Figure \ref{fig:back} shows a back view. The two-piece enclosure snaps
        together through two joints and can easily be taken apart. The
        cylindrical nose must be glued on to the front of the enclosure.}
        \label{fig:cad_enclose}
    \end{figure}

The cylindrical nose-tube is also designed to fit a filter, such as the FR
RG780. When the enclosure was first designed, the \_\_\_ filter was used in the
experiment. When these optimizations are applied to the basic experimental
set-up shown in figure \ref{fig:spcm_setup_basic}, we observe the results shown
in figure \ref{fig:scatter_reduce}.
\begin{figure}[H]%
    \centering
    \begin{subfigure}{.3\textwidth}
    \includegraphics[width=5cm]{ch1_scatter.png}
    \caption{ }
    \label{fig:1}
    \end{subfigure}
    \begin{subfigure}{.3\textwidth}
    \includegraphics[width = 5cm]{ch2_scatter.png}
    \caption{ }
    \label{fig:2}
    \end{subfigure}
    \begin{subfigure}{.3\textwidth}
    \includegraphics[width = 5cm]{cc_scatter.png}
    \caption{ }
    \label{fig:cc}
    \end{subfigure}
    \caption{Figure \ref{fig:1} shows the difference between the Poissonian
    distribution of photon counts per integration time-step in detector A with
    and without scattering reduction hardware. Figure \ref{fig:2} shows the
    difference between the Poissonian distribution of photon counts per
    integration time-step in detector B with and without scattering reduction
    hardware. Figure \ref{fig:cc} shows the difference between the Poissonian
    distribution of photon counts per integration time-step in A-B detector
    coincidences with and without scattering reduction hardware.}
    \label{fig:scatter_reduce}
\end{figure}

We observe that the scattering reduction steps that have been taken cut down on
the Poissonian mean of photon arrivals in channel A and B without a major
reduction in A--B coincidences. This suggests that the optimizations are cutting
out background nose but are not affecting the coincidences generated from
down-converted photons (the signal). 

\subsection{Filters}

Filters that only enable light in the IR range to pass through are effective in
reducing noise, as our laser operates in the IR range. At the beginning of the
experiment, we utilized the filters of type $abcd$. An easy optimization we can
make to the experiment is to improve the quality of the filter. We test the FR
RG780 in place of the $abcd$. The FR RG780 is a longpass filter with a filter
profile that is shown in figure \ref{fig:filter_prof}.
\begin{figure}[H]
    \centering
    \includegraphics[width = 8cm]{COLOR_FILT_XMIT_8_600w.png}
    \caption{Filter profile of RG780 shown in green.}
    \label{fig:filter_prof}
\end{figure}

We compare the results of the experiment with no filter, with the old filters,
and with the new RG780 filters. We perform tests with the purple laser, a
maglite torch, and an IR LED array.

\subsection{Laser}
For the laser, when analyzing initial counts over time, we observe the results
shown in figure \ref{fig:filter_run}.
\begin{figure}[H]%
    \centering
    \begin{subfigure}{.4\textwidth}
    \includegraphics[width=6cm]{filter_ch1_laser.png}
    \caption{ }
    \label{fig:lch1}
    \end{subfigure}
    \begin{subfigure}{.4\textwidth}
    \includegraphics[width = 6cm]{filter_cc_laser.png}
    \caption{ }
    \label{fig:lcc}
    \end{subfigure}
    \caption{Figure \ref{fig:lch1} displays the count numbers measured over each
    integration timestep for a set-up with no filters, old filters in both the
    nose tube and following the output optical fiber, and the RG780 filter in
    both the nose tube and following the output optical fiber. Figure
    \ref{fig:lcc} shows the coincidences measured in each configuration.}
    \label{fig:filter_run}
\end{figure}
We observe that the variance in counts go down in accordance to 1 over the shot
noise, and we see that the RG780 filter outperforms the current filter in
regards to attenuation of the measured count numbers. The coincidence counts
also decrease.

Another useful metric we can examine is the $\alpha$ value for each configuration.
$\alpha$ is the anticorrelation parameter. For two detectors and a channel for coincidence measurement,
we have
\begin{equation}
    \alpha_{2d} = \frac{R_c}{R^{(2r)}_{acc}}.
\end{equation}
Here, $R_{acc}$ denotes the rate of accidental coincidences and is defined to be
\begin{equation}
    R_{acc} = 2\tau R_1 R_2
\end{equation}
where $\tau$ is the coincidence window and $R_1$ and $R_2$ denote the count rates
for channel 1 and channel 2. $R_c$ denotes the coincidence rate. For correlated sources, such as photon pairs arriving from SPDM, we expect $\alpha > 1$.
For an uncorrelated source, such as white light from a torch, we expect $\alpha = 1$.

For each of the filter configurations, we can calculate some $\alpha$ value, finding the
results shown in table \ref{tab:alpha_laser}.
\begin{table}[H]
    \centering
    \begin{tabular}{|l|l|}
    \hline
    \textbf{Configuration} & \textbf{$\alpha$ Value} \\ \hline
    No Filters             & 0.645                   \\ \hline
    Current Filters        & 0.670                   \\ \hline
    RG780                  & 0.764                   \\ \hline
    \end{tabular}
    \caption{Table of $\alpha$ values for various filter configurations with a laser photon source.}
\end{table}\label{tab:alpha_laser}
Curiously, for the laser source, we do not observe an $\alpha$ that is greater than 1, which is
what we would expect for a correlated photon source. This defect will be addressed in section \ref{sec:ccm_broke}.
However, with the new RG780 filter, we achieve a higher $\alpha$ value, indicating that the accidental coincidences
from the experiment are decreases and we observe stronger correlation. This suggests that the RG780 filter outperforms
our current filters.

\subsection{Torch}
We perform the same set of tests described above with a maglite torch. The torch serves
as an uncorrelated photon source, and sends light outwards in a conal shape. We additionally test
this configuration to understand how well the new filter performs in the presence of a source
that primarily emits photons at a wavelength that is attenuated by the filter.

We once again examine the counts over time and the coincidences measured, though in this test
we omit the baseline filter-less case as the counts coming in are too high and excessively compress
the axis of our plots. We observe the results shown in figure \ref{fig:torch_filters}.
\begin{figure}[H]%
    \centering
    \begin{subfigure}{.4\textwidth}
    \includegraphics[width=6cm]{filter_ch1_torch.png}
    \caption{ }
    \label{fig:tch1}
    \end{subfigure}
    \begin{subfigure}{.4\textwidth}
    \includegraphics[width = 6cm]{filter_cc_torc.png}
    \caption{ }
    \label{fig:tcc}
    \end{subfigure}
    \caption{Figure \ref{fig:tch1} displays the count numbers measured over each
    integration timestep for a set-up with old filters in both the
    nose tube and following the output optical fiber and the RG780 filter in
    both the nose tube and following the output optical fiber. Figure
    \ref{fig:tcc} shows the coincidences measured in each configuration.}
    \label{fig:torch_filters}
\end{figure}
Once again, we witness an attenuation in counts and in coincidences. For the case of the
torch, we observe the $\alpha$ values shown in table \ref{tab:alpha_torch}.
\begin{table}[H]
    \centering
    \begin{tabular}{|l|l|}
    \hline
    \textbf{Configuration} & \textbf{$\alpha$ Value} \\ \hline
    Current Filters        & 0.638                   \\ \hline
    RG780                  & 0.677                   \\ \hline
    \end{tabular}
    \caption{Table of $\alpha$ values for various filter configurations with a torch photon source.}
\end{table}\label{tab:alpha_torch}
Interestingly, we once again observe results that contradict theory in that our $\alpha$ value
is not close to 1. However, we do once again observe that the RG780 filters outperform the current filters
in reducing the number of accidental coincidences and yielding a higher $\alpha$ value.

\subsection{IR LED Array}
We also examine the performance of our filters when an IR LED array is placed in front of
our detectors. This varies from the torch in that the wavelength of photons emitted by
the IR LED is a range the filter does not attenuate but the photon source once again only produces
uncorrelated photons. We set-up our IR LED array in accordance with the schematic shown in figure \ref{fig:led_scheme}.
\begin{figure}[H]
    \centering
    \includegraphics[width = 8cm]{IR_LED.jpg}
    \caption{Schematic of IR LED array}
    \label{fig:led_scheme}
\end{figure}
We perform the same measurements as those in the above sections and once again omit
the filterless detector due to high counts. We observe the results shown in figure \ref{fig:ir_comp}.
\begin{figure}[H]%
    \centering
    \begin{subfigure}{.4\textwidth}
    \includegraphics[width=6cm]{filter_ch1_IRled.png}
    \caption{ }
    \label{fig:irch1}
    \end{subfigure}
    \begin{subfigure}{.4\textwidth}
    \includegraphics[width = 6cm]{filter_cc_irled.png}
    \caption{ }
    \label{fig:ircc}
    \end{subfigure}
    \caption{Figure \ref{fig:irch1} displays the count numbers measured with the IR LED array set-up over each
    integration timestep for a set-up with old filters in both the
    nose tube and following the output optical fiber and the RG780 filter in
    both the nose tube and following the output optical fiber. Figure
    \ref{fig:ircc} shows the coincidences measured in each configuration.}
    \label{fig:ir_comp}
\end{figure}

Interestingly, we observe that for the IR LED array, we observe a higher quantity of counts
with the RG780 than with the old filters. This suggests that the RG780 filter attenuates
the actual signal less, and has a steeper slope at the cutoff 780nm frequency. 

When we examine the $\alpha$ values for this set-up, we observe the values shown in
table \ref{tab:ir_cc}.
\begin{table}[H]
    \centering
    \begin{tabular}{|l|l|}
    \hline
    \textbf{Configuration} & \textbf{$\alpha$ Value} \\ \hline
    Current Filters        & 0.66                   \\ \hline
    RG780                  & 0.63                   \\ \hline
    \end{tabular}
    \caption{Table of $\alpha$ values for various filter configurations with a torch photon source.}
\end{table}\label{tab:ir_cc}

Here, we observe that the RG780 filter yields a lower $\alpha$ value than that of the 
current filter, suggesting that it measures more accidental coincidences than the
current filters. Though this is a negative, the strong cutoff and the high transmission of
photons within the signal bandwidth demonstrates that the RG780 is a superior filter than the current filter.

\section{Defects in Current Coincidence Counting Module (CCM)} \label{sec:ccm_broke}

In the previous section, we observe a set of measured $\alpha$ values that do not align with
theoretical results, namely the measurement of uncorrelated light yields $\alpha$s on the order of $10^{-1}$
and the measurement of correlated sources does not yield $\alpha > 1$.

We believe that this result is due to defects in the current coincidence counting module. To further
investigate this result, we analyze the pulse-shaper and the channel-by-channel variance of the counter
to better understand why this error may arise.

\subsection{Coincidence Windows and the Pulse Shaper}

We first analyze the pulse shaper in the CCM. The pulse shaper alters the waveform of a
photon detection event, decreasing the pulsewidths (equivalent to the parameter $\tau$) to those shown in table \ref{tab:pulsewidths}.
\begin{table}[H]
    \centering
    \begin{tabular}{|l|l|}
    \hline
    \textbf{Pulsewidth Setting} & \textbf{Pulsewidth} \\ \hline
    Short        & 10 $\pm$ 2.5ns                   \\ \hline
    Medium        & 14 $\pm$ 2.7ns                   \\ \hline
    Long        & 0.18 $\pm$ 2.8ns                   \\ \hline
    Unaltered                  & 25ns                   \\ \hline
    \end{tabular}
    \caption{Table of $\alpha$ values for various filter configurations with a torch photon source.}
\end{table}\label{tab:pulsewidths}
Coincidences are measured by examing waveform overlaps between two
detectors, and thus a decrease in pulsewidth may lead to a decrease in accidental coincidences,
as two waveforms must arrive within a smaller increment of time for overlap,
and subsequently coincidence to occur.

We investigate the $\alpha$ values yielded for different pulsewidths.
\begin{table}[H]
    \centering
    \begin{tabular}{|l|l|}
    \hline
    \textbf{Pulsewidth Setting} & \textbf{Pulsewidth} \\ \hline
    Short        & 0.653                   \\ \hline
    Medium        & 0.755                   \\ \hline
    Long        & 0.748                   \\ \hline
    Unaltered                  & 0.930                   \\ \hline
    \end{tabular}
    \caption{Table of $\alpha$ values for various filter configurations with a torch photon source.}
\end{table}\label{tab:pulsewidths}
\subsection{Channel Variance}
\section{Altera DE-2 FPGA CCM}

% -------------------------------------------------------------------------------------
% REFERENCES
% -------------------------------------------------------------------------------------
%\bibliography{}

% -------------------------------------------------------------------------------------
% END DOCUMENT
% -------------------------------------------------------------------------------------
\end{document}